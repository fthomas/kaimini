\documentclass[a4paper,12pt]{article}

\usepackage[utf8x]{inputenc}
\usepackage{a4wide}
\usepackage{fancyvrb}
\usepackage[colorlinks=true]{hyperref}

\newcommand{\mailto}[1]
{\href{mailto:#1}{#1}}

\title{\textbf{Kaimini}}
\author{
  Frank S. Thomas\\
  \texttt{\mailto{fthomas@physik.uni-wuerzburg.de}}\\
  \vspace{1em}\\
  Institut für Theoretische Physik und Astrophysik, Universität Würzburg,\\
  Am Hubland, D-97074 Würzburg, Germany
}

\begin{document}

\maketitle

\begin{abstract}
\end{abstract}

\tableofcontents

\section{Introduction}

\cite{Skands:2003cj}

\section{Input Blocks}

\subsection*{\tt{BLOCK KaiminiParameters}}

\subsection*{\tt{BLOCK KaiminiDataPoints}}

\subsection*{\tt{BLOCK KaiminiCalculator}}

\section{Output Blocks}

\subsection*{\tt{BLOCK KaiminiParametersOut}}

\subsection*{\tt{BLOCK KaiminiDataPointsOut}}

\subsection*{\tt{BLOCK KaiminiChiSquare}}

\subsection*{\tt{BLOCK KaiminiCovarianceMatrix}}

\subsection*{\tt{BLOCK KaiminiMinosErrors}}

\subsection*{\tt{BLOCK KaiminiInfo}}


\appendix

\section{Example}

\begin{Verbatim}[frame=single]
Block KaiminiParameters
# <no>  <name>   <use> 
 1  eps_1   RVKAPPAIN;1;1   1   5%  :
 2  eps_2   RVKAPPAIN;2;1   0   5%  :
 3  eps_3   RVKAPPAIN;3;1   1   5%  :
 4  v_L1    RVSNVEVIN;1;1   0   5%  :
 5  v_L2    RVSNVEVIN;2;1   0   5%  :
 6  v_L3    RVSNVEVIN;3;1   0   5%  :
Block KaiminiDataPoints
 1  m^2_atm     SPhenoRP;7;1    1   2.4E-03     0.6E-03
 2  m^2_sol     SPhenoRP;8;1    1   8.0E-05     0.5E-05
 3  tan^2_atm   SPhenoRP;9;1    0   1.0         0.6
 4  tan^2_sol   SPhenoRP;10;1   0   0.45        0.08
\end{Verbatim}

\pagebreak
\bibliographystyle{utphys}
\bibliography{kaimini}

\end{document}
